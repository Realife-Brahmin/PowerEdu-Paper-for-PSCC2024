% \documentclass[class=IEEEtran4PSCC]{standalone}
% \documentclass[article]{standalone}
\documentclass{article}
\ifdefined\packageName
\else
    \newcommand{\packageName}[1]{PowerEdu.jl}
\fi

\begin{document}
    We introduce PowerEdu.jl, an open-source, beginner-friendly package in the Julia
    programming language designed for budding power system engineers. This package 
    addresses the current gap in accessible and comprehensive tools for transmission network
    computations. PowerEdu.jl covers Power Flow using innovative dense 
    and sparse data structures, Continuation Power Flow, State Estimation, Optimal 
    Power Flow, Small-Signal Stability, and Transient Stability Analysis. Notably, 
    the package is scalable, allowing for analysis of systems of varying 
    sizes. User interaction with component modules is highly customizable; for example, 
    users can opt to print detailed intermediary steps, such as Jacobians and mismatches, 
    in Power Flow calculations. We use DataFrames for intuitive and visually 
    appealing data representation. In this paper, we detail the key modules of 
    PowerEdu.jl, elaborate on the special data structures implemented, and demonstrate 
    the breadth and flexibility of algorithm customization available to users. Our 
    package has been rigorously validated against established benchmarks, affirming 
    its reliability and effectiveness as a powerful training tool for the next generation 
    of power system engineers. Mention the network and key finding in one line.

    Why Julia?

\end{document}

