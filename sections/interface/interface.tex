\documentclass[varwidth]{standalone}

\providecommand{\packageName}[1]{PowerEdu.jl}
\providecommand{\packageNameNoJL}[1]{PowerEdu}
\providecommand{\powerflow}[1]{Power Flow}
\providecommand{\sparse}[1]{Sparse Power Flow}
\providecommand{\cpf}[1]{Continuation Power Flow}
\providecommand{\se}[1]{State Estimation}
\providecommand{\opf}[1]{Optimal Power Flow}

\usepackage{lipsum}
\usepackage{dirtree}

\begin{document}

\section{User Interface}

Upon donwloading \packageName{} on their machine, user will interact with the following directory heirarchy. For the sake of clarity, folders pertaining only to the IEEE\_14 Bus test case are shown, however, in general, every test case will have its dedicated folders for inputs and outputs.

\subsection{Directory Structure}

\dirtree{%
.1 {root (\packageNameNoJL{})}.
.2 {data}.
.3 {IEEE\_14}.
.4 {IEEE\_14\_Data.txt}.
% .3 {$\ldots$ (remaining test cases)}.
.2 {processedData}.
.3 {IEEE\_14}.
.4 {BusDataCard\_pu.csv}.
.4 {BranchDataCard\_pu.csv}.
.4 {YBus.csv}.
.4 {$\ldots$ (other generated files)}.
% .3 {$\ldots$ (remaining test cases)}.
.2 {src}.
.3 {ContinuationPowerFlow.jl}.
% .3 {HelperFunctions.jl}.
.3 {IEEE\_CDF\_Parser.jl}.
% .3 {JacobianBuilder.jl}.
% .3 {LU\_Factorization.jl}.
.3 {OptimalPowerFlow.jl}.
.3 {PowerFlow.jl}.
.3 {SparsePowerFlow.jl}.
.3 {StateEstimation.jl}.
.3 {$\ldots$ (other modules)}.
% .3 {YBusBuilder.jl}.
.2 {main.jl}.
.2 {README.md}.
.2 {LICENSE}.
}

\subsection{Pluto Notebook Environment}

\end{document}