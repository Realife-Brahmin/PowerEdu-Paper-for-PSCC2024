\documentclass[varwidth]{standalone}


\providecommand{\packageName}[1]{PowerEdu.jl}


\usepackage{lipsum}

\begin{document}
% \begin{minipage}{0.8\linewidth}
% minipage is good for using the align environment, which otherwise does not 
% like the standalone environment. But it seems to introduce unnecessary 
% space in the main.tex file, in this case, after the abstract. 
% Ideally should test/check why that happens.
% For now, disabling minipage and hoping that nothing bad happens.
\section{Introduction}
Power System Analysis of transmission systems requires competencies in
a variety of aspects. Power System Analysis, which has traditionally associated with quasi-steady
state studies, encompasses aspects such as Power Flow,
Continuation Power Flow, Economic Dispatch, Optimal Power Flow,
State Estimation and so on. Each of these aspects requires a broad knowledge of
Mathematics and Physics but also a wide variety of skillsets, including
strong programming skills, in order to actually test novel algorithms
and bringing essential control schemes to fruition. There can be a significant
gap between understanding of the theory and actual implementation for Power
System studies and day-to-day operation. Aspects like Sparse Power Flow
which requires usage of special data structures of the same name especially highlight
how actual implementation can vary from textbook algorithms, which are often
written in pseudo code. While academic curricula and textbooks provide a strong
foundation in theory, there remains a pressing need for practical tools that
translate these theoretical underpinnings into tangible, implementable solutions.
This gap becomes particularly evident when students or new professionals are
tasked with the direct application of these theories in real-world scenarios.
While various open-source packages exist to aid power system researchers, many are designed with a primary focus on delivering end results. These tools may not be as accommodating for newcomers who are eager to understand the underlying processes, delve deep into the intricacies of algorithms, or get their hands dirty with the code. Our software acknowledges this gap. Recognizing the educational journey many newcomers embark upon, we've crafted our package to not only deliver accurate results but also facilitate a deeper understanding. Users can easily access internal variables during iterative processes, something many other packages shield away. Furthermore, \packageName{} stands out for its high level of customizability, allowing individuals to tinker, modify, and adapt algorithms to their specific needs, and thus offers a unique, hands-on experience.
Our free and open-source package, \packageName{} aims to serve as a
bridge for budding power system engineers who
may find the initial stages of coding and computational analysis challenging.
By offering an accessible, well-documented and easy to tinker platform, we
aim to narrow the gap between newcomers to the field and seasoned experts
who have dedicated years at renowned national laboratories or corporations,
developing sophisticated software tools utilized
by the industry.
% \end{minipage}
\end{document}
